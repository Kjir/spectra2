\documentclass[a4paper,11pt,twoside,openright]{unibo}

\usepackage{newlfont}
\usepackage{url}
\usepackage[backref]{hyperref}
\textwidth=450pt\oddsidemargin=0pt

\begin{document}
\title{Sviluppo di un software per l'analisi real-time di dati
radioastronomici su macchine multicore}
\author{St\'ephane Bisinger}
\date{May 2010}
\acyear{2009 --- 2010}
\supervisor{Antonella Carbonaro}
\thsubject{Programmazione}
\thsession{Prima}
\pagenumbering{roman}

\maketitle
%\include{frontpage}

\tableofcontents

\chapter*{Introduction}
\label{intro}
\addcontentsline{toc}{chapter}{Introduction}
\chapter{Mathematical background}
\pagenumbering{arabic}
\label{math_bkg}
In this chapter we will introduce some basic informations about Digital Signal
Processing necessary to understand the purpose of a spectrometer and its field
of application. A basic understanding of what a signal is, how we can extract
and elaborate information from it, the mathematical background and its
complexity is required for the comprehension of the inner workings of the
spectrometer and of the problems we faced during its development.\\
This is introductory material sufficient to have an overview of
the theory involved, for further details please refer to the
bibliography.
\section{Signals}
\section{ADC: Analog--to--Digital Conversion}
\section{The Fourier Transform}
\section{The Discrete Fourier Transform}
\section{FFT: Fast Fourier Transform}
Taken from \cite{bertoni}
\chapter{Project outline}
\label{outline}
\chapter{Software development}
\label{sw_devel}
See \ref{tests}.
\chapter{Test results and observations}
\label{tests}
\chapter{Conclusions}
\label{conclusions}
\bibliographystyle{alpha}
\bibliography{thesis}
\end{document}
