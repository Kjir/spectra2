\documentclass[a4paper,11pt,twoside,openright]{unibo}

\usepackage[italian]{babel}
\usepackage{newlfont}
\usepackage{url}
\usepackage[backref]{hyperref}
\textwidth=450pt\oddsidemargin=0pt

\begin{document}
\title{Sviluppo di un software per l'analisi real-time di dati
radioastronomici su macchine multicore}
\author{St\'ephane Bisinger}
\date{May 2010}
\acyear{2009 --- 2010}
\supervisor{Antonella Carbonaro}
\thsubject{Programmazione}
\thsession{Prima}
\pagenumbering{roman}

\maketitle
%\include{frontpage}

\tableofcontents

\chapter*{Introduzione}
\label{intro}
\addcontentsline{toc}{chapter}{Introduzione}
\chapter{Basi di elaborazione del segnale}
\pagenumbering{arabic}
\label{math_bkg}
In quest capitolo introdurremo alcuni concetti basilari sull'elaborazione dei
segnali numerici necessari alla comprensione del funzionamento di uno
spettrometro e del suo campo di applicazione. Sapere cosa sia un segnale, come
si estraggono ed elaborano le informazioni in esso contenuto, quali siano i
concetti matematici utilizzati è fondamentale per capire il lavoro svolto.\\
Questo materiale introduttivo è sufficiente per avere una panoramica sui
concetti teorici utilizzati, per approfondimenti fare riferimento alla
bibliografia.
\section{Segnali}
Il nostro corpo è in grado di percepire molte delle variazioni che avvengono nel mondo che ci circonda: 
\section{ADC: Analog--to--Digital Conversion}
\section{La Trasformata di Fourier}
\section{La Trasformata Discreta di Fourier}
\section{FFT: Fast Fourier Transform}
Taken from \cite{bertoni}
\chapter{Progettazione e architettura}
\label{outline}
\section{Obiettivi}
\section{Gerarchia di classi}
\chapter{Sviluppo del progetto}
\label{sw_devel}
\section{Librerie}
See \ref{tests}.
\chapter{Risultati e considerazioni}
\label{tests}
\chapter{Conclusioni e sviluppi futuri}
\label{conclusions}
\bibliographystyle{alpha}
\bibliography{thesis}
\end{document}
