\chapter{Basi di elaborazione del segnale}
\pagenumbering{arabic}
\label{math_bkg}
In questo capitolo introdurremo alcuni concetti basilari sull'elaborazione dei
segnali necessari alla comprensione del funzionamento di uno
spettrometro e del suo campo di applicazione. Sapere cosa sia un segnale, come
si estraggono ed elaborano le informazioni in esso contenuto, quali siano i
concetti matematici utilizzati \`e fondamentale per capire il lavoro svolto.\\
Questo materiale introduttivo \`e sufficiente per avere una panoramica sui
concetti teorici utilizzati.

\section{Segnali}
Il nostro corpo \`e in grado di percepire, attraverso i sensi, alcune delle
variazioni nelle propriet\`a fisiche del mondo che ci circonda. Queste variazioni che percepiamo vengono
elaborate dal nostro cervello che riesce a ricavarne informazioni utili: In questo modo
siamo in grado di percepire se fa caldo o freddo, se c'\`e luce, se un oggetto
\`e di un colore piuttosto che un altro, ecc. Va notato che in questi segnali
c'\`e una componente fisica (la temperatura) e l'informazione veicolata
(freddo/caldo). \cite{bertoni}
\begin{definitions} \label{def:signal}
Si dice segnale una qualunque quantit\`a che varia nel tempo o nello spazio.
\end{definitions}
Secondo la precedente definizione, quindi, \`e possibile che un segnale non
contenga informazioni utili: in questo caso ci si riferisce al segnale come
\emph{rumore}. Il rumore \`e anche una componente di interferenza o errore su di
un segnale che si cerca di interpretare.

Un segnale viene quindi rappresentato da una funzione $g = f(c)$, ove:
\begin{itemize}
    \item $g$ \`e una variabile dipendente su di una grandezza fisica in
    relazione al tempo o spazio.
    \item $c$ \`e una varaibile indipendente che rappresenta lo spazio o il
    tempo.
    \item $f$ \`e una funzione che associa ad un valore temporale o spaziale $c$
    la corrispondente quantit\`a $g$.
\end{itemize}

Per semplicit\`a di esposizione, d'ora in avanti si far\`a riferimento
unicamente a segnali in relazione al tempo.
\subsection{Segnali periodici}
Una importante classe di segnali sono i segnali periodici. Un segnale \`e
periodico se si ripete in un certo periodo di tempo: definendo $T$ il periodo con
cui si ripete il segnale e con $t$ un qualunque momento nel tempo, per ogni $k$,
vale $f(t) = f(t + kT)$. Un segnale periodico $f(t)$ \`e quindi univocamente
individuato nell'intervallo $\frac{T}{2} \le t \le \frac{T}{2}$.

Si definisce  \emph{frequenza} di un segnale periodico $f(t)$ il numero di
ripetizioni del periodo $T$ nell'unit\`a di tempo. Risulta quindi:
\[
\nu = \frac{1}{T}
\]
L'unit\`a di misura della frequenza \`e l'Hertz ed indica il numero di cicli
(ripetizioni) al secondo.
\subsection{Frequenza radio}
\section{ADC: Analog--to--Digital Conversion}
Per interpretare i segnali del mondo che ci circonda, possiamo sfruttare la
velocit\`a di calcolo di un elboratore; tuttavia per poterlo usare, bisogna
essere in grado di rappresentare le informazioni che riceviamo in modo che
l'elaboratore sia in grado di capirle. Per questo motivo dobbiamo trovare un
modo di trasformare un segnale \emph{analogico} in un segnale \emph{digitale}.
\subsection{Segnali analogici, segnali digitali}
Come abbiamo visto nella definzione \ref{def:signal}, i segnali hanno due
componenti: il tempo e il valore assunto dalla grandezza fisica che osserviamo.
Queste due componenti possono assumere un qualunque valore reale, cio\`e il
segnale ha \emph{tempo continuo} e \emph{valori continui}. Tuttavia, in questo
modo il segnale non pu\`o essere rappresentato da un elaboratore, in quanto un
elaboratore pu\`o contenere solamente quantit\`a finite, mentre le componenti
del segnale sono quantit\`a infinite. Per questo motivo bisogna cercare di far
rientrare il segnale in un range di valori finiti e renderlo cos\`i
rappresentabile da un calcolatore.

Per rendere finita la misurazione del tempo, si possono salvare i valori
rilevati ad un intervallo prefissato, ad esempio una volta ogni secondo.
Misurando un segnale per 10 secondi si otterranno cos\`i 10 valori associati ad
ogni secondo. Un segnale di questo tipo ha \emph{tempo discreto}, ma valori
ancora \emph{continui}. Un dispositivo che raccoglie valori con una determinata
frequenza si chiama \emph{campionatore}.

Per rendere i valori finiti, si pu\`o scegliere di utilizzare un certo insieme
di grandezze, ad esempio $1$ e $-1$, e per ogni valore rilevato, associare la
grandezza pi\`u vicina. Quindi se rileviamo i valori $4$, $10$, $-5$ e $-13$,
salviamo il segnale con i valori $1$, $1$, $-1$, $-1$. Cos\`i facendo, il
segnale assume \emph{valori finiti} che sono rappresentabili in un elaboratore,
nell'esempio fatto con l'uso di 1 bit. Un dispositivo che associa ai valori
rilevati il pi\`u vicino valore rappresentabile dall'elaboratore si chiama
\emph{quantizzatore}.

Un segnale che ha tempo \emph{continuo} e valori \emph{continui} si chiama
\emph{segnale analogico}, mentre un segnale con tempo \emph{discreto} e valori
\emph{finiti} si chiama \emph{segnale digitale}.

Come \`e facile intuire, la conversione da analogico a digitale pu\`o introdurre
degli \emph{errori}, cio\`e del \emph{rumore}, in quanto la versione digitale di
un segnale analogico \`e una approssimazione. Fortunatamente \`e possibile
valutare questo margine di errore e ridurlo a seconda delle necessit\`a sia
aumentando il numero di bit usati per rappresentare i valori nel tempo, sia
aumentando il numero di rilevazioni effettuate nello spazio temporale di
osservazione.
\subsection{Teorema del Campionamento}
\section{La Trasformata di Fourier}
\section{La Trasformata Discreta di Fourier}
\section{FFT: Fast Fourier Transform}
Taken from \cite{bertoni}
