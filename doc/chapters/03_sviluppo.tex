\chapter{Sviluppo del progetto}
\label{sw_devel}

Durante lo sviluppo del progetto si ha fatto uso di librerie e strumenti
software per semplificare il pi\`u possibile il lavoro da svolgere ed ottenere
il massimo delle prestazioni possibili. 

\section{Librerie}
L'utilizzo delle giuste librerie permette di sviluppare pi\`u velocemente i
propri progetti, ottenere prestazioni spesso migliori rispetto ad una soluzione
sviluppata in proprio e sicuramente avere pi\`u certezze riguardo al corretto
funzionamento delle parti attenenti alla libreria.

\subsection{Boost}
Le librerie Boost\footnote{\url{http://www.boost.org/}} sono molto usate nella
programmazione in \CC perch\'e implementano moltissime funzioni di comune
utilizzo fornendo una interfaccia molto semplice. L'ottima qualit\`a di queste
librerie \`e confermato dal fatto che spesso alcune sue componenti vengono
integrate nello standard \CC. Queste librerie sono basate sulla versione ANSI
del \CC, rendendole compatibili con qualunque compilatore che implementi gli
standard ANSI.In questo progetto vengono utilizzate principalmente per
semplificare la lettura di argomenti da linea di comando, la lettura/scrittura
di files, le comunicazioni di rete e il multi-threading.

\subsection{\ac{ipp}}
Le Integrated Performance
Primitives\footnote{\url{http://software.intel.com/en-us/intel-ipp/}} sono delle
librerie che implementano un insieme molto completo di funzioni e algoritmi
relativi al processamento di segnali e crittografia. In particolare implementano
funzioni utili al processamento di segnali audio (filtri, codifiche audio,
compressione dei dati, ecc.), funzioni per il processamento di immagini
(trasformazioni, codifica video, ecc.), funzioni per calcoli su matrice e
funzioni crittografiche (crittografia simmetrica, asimmetrica, funzioni hash,
ecc.).  Il vantaggio delle IPP \`e che offrono un ampio spettro di funzioni che
coprono praticamente tutte le necessit\`a nel processamento dei segnali, oltre
ad essere ottimizzate per processori Intel. Inoltre sono state scritte in modo
da essere naturalmente utilizzabili in ambienti multi-threaded, sfruttando tutte
le potenzialit\`a di processori multicore o processori multipli. Tutti questi
fattori hanno concorso nell'adozione di questa libreria al posto di altre
librerie concorrenti. Questa libreria \`e l'unico componente non Open Source
utilizzato nell'intero progetto, anche se il suo utilizzo \`e gratuito per usi
accademici. Le librerie alternative potrebbero essere comparate per verificare
l'effettiva qualit\`a a livello di prestazioni\footnote{cfr. paragrafo
\ref{altlib}}.

\subsubsection{Multithreading e prestazioni nelle \ac{ipp}}
Le \ac{ipp} sono state sviluppate con l'intenzione di sfruttare a pieno le
istruzioni \ac{SIMD} e \ac{SSE} presenti nei moderni processori. Queste
istruzioni permettono di effettuare operazioni su vettori in parallelo,
guadagnando grandi vantaggi prestazionali in alcuni tipi di operazioni, tra cui
anche i calcoli necessari per l'elaborazione di segnali. Siccome diversi
processori supportano diversi tipi di istruzioni, possono esistere diverse
implementazioni di alcune funzione in base al tipo di istruzioni supportate. Per
selezionare automaticamente la funzione corretta, le \ac{ipp} dispongono di un
dispatcher che, durante l'inizializzazione a run-time, individuano le capacit\`a
del processore e quindi la categoria di libreria da utilizzare. Questo permette
di usare trasparentemente le funzioni di libreria sfruttando sempre
l'implementazione pi\`u efficiente per il processore utilizzato.

Alcune funzioni di base della libreria sfruttano il threading per migliorare le
proprie prestazioni, in particolare sfruttando le librerie OpenMP sviluppate da
Intel. Quando si utilizza la versione multi-threaded della libreria, i thread
vengono automaticamente inizializzati al numero di processori e core presenti: a
questo modo la libreria avvier\`a al massimo tanti thread quante sono le unit\`a
di calcolo presenti nel sistema, evitando di sovracaricare il sistema e
ottenendo cos\`i la migliore prestazione possibile dal sistema in uso. Sono
presenti anche due funzioni, \texttt{ippSetNumThreads()} e
\texttt{ippGetNumThreads()}, che permettono di manipolare il numero di threads
utilizzati dalla libreria. Il numero di thread non sar\`a mai maggiore rispetto
al numero di unit\`a di calcolo del sistema, anche se impostati manualmente ad
un valore maggiore.

Tutte le funzioni IPP sono thread-safe, cio\`e possono essere usate da diversi
thread di esecuzione contemporaneamente senza che questo implichi un qualunque
malfunzionamento.\footnote{Per approfondimenti, cfr.
\url{http://software.intel.com/en-us/articles/threading-and-intel-integrated-performance-primitives/}}


\section{Codice pre-esistente}
Parte del codice del progetto \`e stato prelevato dal software sviluppato in
fase di tirocinio, sempre presso l'Istituto di Radioastronomia di Medicina.
Questo codice ha permesso di ridurre notevolmente il tempo dedicato allo
sviluppo del calcolo della \ac{FFT} lasciando maggior tempo per una corretta
implementazione delle altre parti. Questo codice \`e stato adattato per
funzionare in un contesto multi-threaded e per accettare input da diverse fonti.

\section{Gestione di problemi comuni}
\subsection{Selezione del tipo di dato}
Per rendere il programma il pi\`u versatile possibile, non si poteva legare il
suo funzionamento ad un unico tipo di dati. Siccome le \ac{ipp} includono
funzioni per diversi tipi di dati, il programma \`e stato sviluppato in modo da
riuscire a selezionare il giusto tipo di funzione a seconda del tipo di dato in
input e in output. L'implementazione attuale permette di selezionare il tipo di
dato impostando delle costanti e ricompilando il codice, un possibile
miglioramento potrebbe essere la selezione dinamica senza bisogno di
ricompilazione del tipo di dato, cui parleremo nel paragrafo \ref{dyn_dtype}.

\subsubsection{Dal numero di bit al tipo di variabile}
Il primo passo da compiere per poter lavorare sul tipo di dato desiderato \`e
associare al numero di bit selezionati nella costante il corretto tipo di dato.
Il listato \ref{lst:consts_def} illustra il metodo di selezione dei numeri di bit e
la definizione delle costanti contenenti i tipi di dato corretti. Come si pu\`o
osservare, si usano i template per ottenere il risultato che vogliamo e con
delle \texttt{typedef} definiamo dei nuovi tipi corrispondenti alle dimensioni
richieste. Parte del codice di \texttt{typer} \`e riportato nel listato
\ref{lst:typer}. Si pu\`o osservare come molto semplicemente al variare dei due
parametri del template, si definisce un tipo denominato \texttt{type}
corrispondente alle caratteristiche richieste.

\lstinputlisting[firstline=25,lastline=34,float,caption=Definizione di costanti
per la selezione del tipo di
dato,label=lst:consts_def,basicstyle=\tiny]{../src/main.cc}

\lstinputlisting[firstline=4,lastline=21,float,caption=La magia dietro
\texttt{typer},label=lst:typer,basicstyle=\small]{../src/type.hpp}

\subsubsection{Gestione della memoria}
La libreria \ac{ipp} fornisce alcune funzioni per allocare memoria
allineata\footnote{Gli elaboratori quando accedono ai dati in memoria, lo fanno
leggendo un certo numero di bytes alla volta. Per ottenere migliori
prestazioni, \`e ideale che la memoria per le variabili venga allocata
allineata al punto in cui il processore inizier\`a a leggerla.} ed effetuare
altre operazioni di basso livello. Siccome la libreria \`e scritta in C, ogni
tipo di dato ha una funzione diversa, quindi per poter scrivere codice generico
che funzioni con qualunque tipo di dato \`e stato necessario creare
un'interfaccia \CC che richiami automaticamente le funzioni corrette. Il metodo
adottato (listato \ref{lst:malloc}) prevede che la funzione generica di
interfaccia prenda due argomenti invece di uno solo: il primo argomento \`e un
puntatore al vettore per cui si sta allocando la memoria ed il secondo
argomento \`e la lunghezza di questo vettore. Il primo argomento in realt\`a
non viene utilizzato, ma con la tecnica dell'\texttt{overloading} permette di
selezionare la funzione corretta.

\lstinputlisting[firstline=108,lastline=130,float,caption=Alcune implementazioni
della funzione
\texttt{IPP::malloc},label=lst:malloc,basicstyle=\small]{../src/ipp.cc}

\subsubsection{Buffer di dati}
Per le classi e strutture contenente i dati \`e stata scelta una strategia
diversa: siccome non \`e necessario chiamare funzioni di libreria, ma i metodi
sono generici ed indipendenti dal tipo di dato su cui si opera, sono stati
adottati i template come soluzione generica. Come si pu\`o vedere nella
definizione della struttura \texttt{SrcType} (listato \ref{lst:srctype}), i
membri della struttura sono assolutamente neutri rispetto al tipo di dato
utilizzato.

\lstinputlisting[firstline=227,lastline=236,float,caption=Dichiarazione della
struttura \texttt{SrcType},label=lst:srctype]{../src/fft_buf.hpp}

\subsection{Client UDP come SourceFilter}
Nel paragrafo \ref{sourcefilter} abbiamo spiegato come in fase di progettazione
si sia pensato ad una interfaccia generica per leggere i dati da una fonte
esterna; il listato \ref{lst:sourcefilter} mostra l'implementazione di questa
interfaccia.
\lstinputlisting[firstline=4,lastline=10,float,caption=Dichiarazione
dell'interfaccia
\texttt{SourceFilter},label=lst:sourcefilter]{../src/filter/source.hpp}

Per implementare questa interfaccia \`e sufficiente ereditare dalla classe
virtuale \texttt{SourceFilter} ed implementare il metodo \texttt{read()}.
Vediamo ad esempio la dichiarazione della classe \texttt{udp\_sock}, presentata
nel listato \ref{lst:udp_sock}: si pu\`o notare che vengono definite due
funzioni read, una delle quali corrisponde a quella dichiarata nella classe
virtuale \texttt{SourceFilter}. L'implementazione della \texttt{read()}
virtuale, presentata nel listato \ref{lst:udp_read}, non fa altro che convertire
il puntatore a void in un puntatore del tipo specificato dal template. Questo
permette di usare la classe \texttt{udp\_sock} con qualunque tipo di dato.

\lstinputlisting[firstline=25,lastline=41,caption=Dichiarazione della classe
\texttt{udp\_sock},float,label=lst:udp_sock,basicstyle=\small]{../src/server.hpp}

\lstinputlisting[firstline=183,lastline=186,caption=Implementazione della
\texttt{read()} virtuale,float,label=lst:udp_read]{../src/server.hpp}

\subsection{Gestione del threading}
Quando si introduce il threading in una applicazione, ci sono delle situazioni
che vanno prese in considerazione: avere diversi thread pu\`o portare problemi
normalmente assenti e non facilmente identificabili. Inoltre alcuni problemi
legati al threading possono essere difficili da riprodurre perch\'e  dipendono
dalla schedulazione dei thread da parte del sistema operativo, quindi sono al di
fuori dal controllo del programmatore. Per questo motivo esistono alcune
tecniche di programmazione che permettono di evitare problemi quando thread
diversi devono condividere alcuni dati.

\subsubsection{Mutua esclusione}
\lstinputlisting[firstline=3,lastline=12,float,caption=Variabili della
lista,label=lst:vars]{../src/list.hpp}
Con la mutua esclusione si assicura che quando un thread accede ad una parte di
dati protetta, nessun altro thread pu\`o accedervi contemporaneamente. La mutua
esclusione \`e stata implementata, ad esempio, come wrapper attorno alla
struttura standard \texttt{std::list} per renderla utilizzabile da pi\`u thread
contemporaneamente.

Per rendere la lista thread-safe, servono tre variabili: una contiene la lista
standard, una per la mutex ed una per la condition variable, come mostrato nel
listato \ref{lst:vars}

Una mutex, abbreviazione di \textbf{mut}utal \textbf{ex}clusion, serve appunto
per poter bloccare l'accesso all'oggetto da parte di altri thread e per
rilasciare il blocco quando si ha finito. Le funzioni presenti in
\texttt{std::list} vengono reimplementate sfruttando la mutex per assicurare un
comportamento corretto in caso di multi-threading: troviamo un buon esempio nel
metodo \texttt{empty()} presentato nel listato \ref{lst:empty}.
\lstinputlisting[firstline=24,lastline=28,float,caption=Implementazione di
\texttt{empty()},label=lst:empty]{../src/list.hpp}

La creazione dell'oggetto di tipo \texttt{boost::mutex::scoped\_lock} sospende
l'esecuzione del thread se ci sono gi\`a altri thread che possiedono l'esclusiva
sull'oggetto \texttt{\_mut}, altrimenti acquisisce il diritto di esclusiva
sulla mutex e procede con l'operazione. Nel distruttore dell'oggetto
\texttt{lock} avviene il rilascio della mutex, quindi non appena l'oggetto
finisce fuori contesto --- cio\`e dopo il \texttt{return} --- il distruttore
viene chiamato e la mutex viene liberata, lasciando libero accesso ad altri
threads.

\subsubsection{Wait/notify}
Abbiamo visto nel paragrafo \ref{threadout} come il thread di output si metta in
attesa sulla lista quando questa \`e vuota. Il codice presente nel thread di
output \`e simile a questo:
\begin{lstlisting}
while(list.empty()) {
    list.wait();
}
//Do something
\end{lstlisting}

Questo codice mette in uno stato ``dormiente'' il thread, lasciando il posto
agli altri thread e ottimizzando cos\`i il tempo di scheduling assegnato al
programma: l'alternativa sarebbe fare un ciclo continuo --- detto \emph{busy waiting} --- per verificare la stessa
condizione ripetutamente, ma in questo modo si sprecherebbero tantissime risorse
su un controllo che non cambier\`a risultato prima che avvengano altre cose. Per
monitorare lo stato di questa condizione, si pu\`o utilizzare una
\texttt{boost::condition\_variable}, come mostrato nel listato \ref{lst:cond_wait}.
\lstinputlisting[firstline=50,lastline=53,float,caption=Uso della
\texttt{boost::condition\_variable},label=lst:cond_wait]{../src/list.hpp}
Il codice inizialmente acquisisce un lock esclusivo sulla mutex, poi si mette in
attesa sulla \texttt{condition\_variable} \texttt{\_empty}. A questo punto
il lock viene rilasciato ed il thread si mette in attesa, senza occupare risorse
sul sistema. Per risvegliare questo thread bisogna notificarlo del fatto che la
condizione su cui stava aspettando `e stata modificata, quindi quando si
inserisce un nuovo elemento nella lista bisogna anche notificare il thread di
output che la lista non \`e pi\`u vuota, come illustrato nel listato
\ref{lst:notify}. Il metodo \texttt{notify\_one()} \`e definito come:
\begin{lstlisting}[frame=none]
    void notify_one() { _empty.notify_one(); }
\end{lstlisting}
Utilizza, quindi, la \texttt{condition\_variable} per notificare uno dei thread
in attesa sulla variabile stessa. A questo punto il thread di output si
risveglia, aspetta di poter riacquisire il lock sull'oggetto e continua con
l'esecuzione. A questo punto la condizione \texttt{list.empty()} dovrebbe
essere falsa e il thread pu\`o procedere con il suo lavoro.
\lstinputlisting[firstline=34,lastline=38,float,caption=Notifica di inserimento
di un nuovo elemento in lista,label=lst:notify]{../src/list.hpp}

\subsubsection{Debugging}
Per capire meglio i problemi che possono sorgere osserviamo il codice nel
listato \ref{lst:threrr}:
\begin{lstlisting}[numbers=left,frame=l,float,caption=Codice problematico nel caso
	di pi\`u threads attivi,label=lst:threrr]
if( list.empty() || list.back()->is_src_full() ) {
	FFTBuf<DstIppType> *buffer( new FFTBuf<DstIppType>(siglen, sums) );
	*buffer = IPP::alloc(buffer->cdata(), siglen);
	IPP::zero_mem(buffer->cdata(), siglen);
	list.push_back(buffer);
}
FFTBuf<DstIppType> *work_buffer = list.back();
\end{lstlisting}
questo codice faceva parte del thread principale ed \`e il controllo effettuato
per verificare se \`e necessario creare un nuovo buffer di output o se si pu\`o
assegnare l'ultimo buffer in lista come punto di scrittura della trasformazione
da effettuare, meccanismo descritto nel paragrafo \ref{main_thread}.
\paragraph{Analisi del codice}
Siccome il listato non \`e banale, procediamo ad analizzare passo per passo le
operazioni effettuate:

Nella riga 1, si controlla innanzitutto se la lista di output \`e vuota o se
l'ultimo elemento nella lista \`e stato assegnato tante volte quanto sono il
numero di somme che si vogliono effettuare. Se almeno una delle due condizioni
\`e vera, significa che dobbiamo accodare un nuovo elemento nella lista. La riga
2 crea il puntatore \texttt{buffer} ed inizializza l'oggetto di tipo
\texttt{FFTBuf<DstIppType>} assegnando come parametri la lunghezza del segnale
elaborato (\texttt{siglen}) ed il numero di somme che si intendono fare su quel
buffer (\texttt{sums}). \texttt{FFTBuf} \`e un oggetto che contiene il buffer in
cui viene salvato il segnale una volta applicate le trasformazioni e alcuni
metodi ausiliari. Uno di questi metodi, usato nella riga 1, \`e
\texttt{is\_src\_full()} che restituisce un valore booleano vero se l'istanza
dell'oggetto \`e stata assegnata ad un numero di thread pari al numero di somme
volute, falso altrimenti. Con questo metodo, quindi, si pu\`o sapere se
l'istanza dell'oggetto \`e riutilizzabile per un altro thread o meno. La
costante DstIppType segnala il tipo di dato su cui si opera, in particolare il
numero di bit usati per ogni punto del segnale: avere questa costante permette
di adattare il programma molto velocemente per tipi di dati diversi. Alle righe
3--4 si alloca la memoria per il buffer, utilizzando i metodi delle \ac{ipp} per
fare s\`i che la memoria sia allineata e la si inizializza a zero.  Nella riga 5
si accoda il puntatore al buffer in quanto ora l'inizializzazione \`e
completata. La riga 7 mostra come viene selezionato il buffer su cui lavorare:
grazie al codice precedente si tratter\`a sempre dell'ultimo buffer inserito
nella lista.

\paragraph{Comportamento in ambiente multi-threaded}
Il codice appena descritto funziona perfettamente se non si utilizza il
threading. Tuttavia, durante la fase di test, il programma funzionava il pi\`u
delle volte, ma presto o tardi l'esecuzione terminava con un errore di
\emph{segmentation fault}\footnote{La segmentation fault \`e un errore tipico
    dei programmi C/\CC. Significa che un programma ha tentato di accedere ad
    una zona di memoria non di sua competenza e il sistema operativo non
    concede di farlo per ovvi motivi di sicurezza. Tipicamente questo problema
    avviene quando si legge oltre la fine di un array o si tenta di accedere ad
    un puntatore non allocato o impostato a NULL.}.
Osservando il problema con l'ausilio di un debugger, il problema non risultava
chiaro perch\'e si presentava in punti differenti del codice e sempre in
situazioni in cui non ci si aspetta di avere un puntatore nullo, rendendo
l'investigazione tramite debugger praticamente inutile. Il fatto che il problema
si mostrasse in modo casuale, in momenti diversi e in punti diversi del codice
lasciava intuire che si trattasse di un problema di concorrenza, quindi
riesaminando l'interazione dei vari thread tra di loro si \`e riusciti a scovare
e risolvere la fonte del problema:  come descritto nel paragrafo
\ref{threadout}, il thread di output lavora sulla lista di buffer e quando
finisce di scrivere i dati sull'output, elimina il buffer dalla lista. Se
osserviamo in particolare la riga 1 del listato \ref{lst:threrr}, vediamo che si
chiede prima se la lista \`e vuota e in seguito si accede all'ultimo elemento in
lista. Ma cosa succede se subito dopo l'istruzione \texttt{list.empty()} avviene
un \emph{context switch} e il thread di output riprende l'esecuzione, scrive
l'ultimo buffer presente il lista sull'output e lo elimina? In questo scenario
la successiva chiamata a \texttt{list.back()} ritornerebbe un puntatore nullo e
la chiamata a \texttt{is\_src\_full()} provoca la segmentation fault. Per
evitare che questo avvenga, bisogna richiedere il lock sulla mutex della lista
di buffer di output prima di testare le condizioni, a questo modo mentre si
verifica questa condizione lo stato della lista non pu\`o essere alterato.

\paragraph{Conclusioni e osservazioni}
La soluzione al problema, presentata nei listati \ref{lst:thr_correct} e
\ref{lst:item_needed} \`e stata raggiunta dopo molto tempo perso a ricercare la
fonte di questo problema causato da una non corretta analisi dei comportamenti
in ottica multi-threaded.  Individuare questo genere di errori \`e molto
difficile, mentre commettere questo tipo di errori \`e molto facile: per questo
motivo \`e generalmente consigliabile evitare la concorrenza a meno che non vi
siano chiari vantaggi nell'adottarla; e quando si decide di sfruttare la
concorrenza nel proprio programma, \`e importante considerare tutti gli scenari
possibili nelle interazioni tra i thread soprattutto quando si effettuano
operazione non atomiche\footnote{Una operazione, ad esempio \lstinline$int i =
    5;$ \`e atomica se la sua esecuzione non pu\`o essere interrotta a met\`a: o
        l'operazione \`e ancora da effettuare, oppure \`e completata. In \CC 
        l'operazione di cui sopra viene tradotta in diverse istruzioni macchina
        e quindi pu\`o avvenire un context switch a met\`a dell'operazione. Le
        istruzioni in \CC, quindi, non sono atomiche. Esistono altri linguaggi
        di programmazione, pensati appositamente per la programmazione
        multi-threaded, che garantiscono l'atomicit\`a delle loro istruzioni.}.
\begin{lstlisting}[float,label=lst:thr_correct,caption=Codice funzionante anche
in ambiente multi-threaded]
if(list.item_needed()) {
	FFTBuf<DstIppType> *buffer( new FFTBuf<DstIppType>(siglen, sums) );
	*buffer = IPP::alloc(buffer->cdata(), siglen);
	IPP::zero_mem(buffer->cdata(), siglen);
	list.push_back(buffer);
}
FFTBuf<DstIppType> *work_buffer = list.back();
\end{lstlisting}
\lstinputlisting[firstline=55,lastline=58,float,caption=Implementazione del
metodo \texttt{item\_needed()},label=lst:item_needed]{../src/list.hpp}

\subsection{Memory leak}
Un altro problema molto diffuso nella programmazione in \CC sono i \emph{memory
leak}. Con questo termine si indicano quelle perdite di memoria che avvengono
perch\'e ad allocazione di memoria da una parte, non corrisponde una
deallocazione da un'altra parte del codice. Se un problema di questo genere
esiste e non viene corretto, un programma che rimane attivo per un certo periodo
di tempo incrementa sempre pi\`u il suo utilizzo di memoria, fino a che la
memoria disponibile nel sistema non viene interamente consumata. A questo punto
non solo il programma non pu\`o pi\`u funzionare, ma anche gli altri programmi
presenti sul sistema possono riscontrare problemi per la mancanza di memoria da
allocare; in alcuni casi il sistema potrebbe diventare addirittura
inutilizzabile e potebbe diventare necessario un riavvio del sistema. Purtroppo,
se la perdita di memoria \`e sufficientemente piccola, \`e facile che un
problema di questo tipo possa sfuggire in fase di test e creare problemi solo
dopo un lungo utilizzo in fase di produzione. Per questo motivo esistono dei
tool, come ad esempio la suite di
Valgrind\footnote{\url{http://www.valgrind.org/}}, che aiutano
nell'individuazione di questi problemi. Durante lo sviluppo del progetto \`e
stato incontrato un errore di questo tipo nell'implementazione del buffer per i
dati di output.

\subsubsection{Implementazione originaria}
La classe \texttt{FFTBuf} contiene un buffer e qualche metodo per gestire
l'accesso al buffer stesso. Nella sua prima versione il codice per
inizializzarlo era il seguente:
\begin{lstlisting}[caption=Inizializzazione di un FFTBuf (vecchia
versione),label=lst:old_fftbuf]
FFTBuf<DstIppType> *buffer( new FFTBuf<DstIppType>(siglen, sums) );
*buffer = IPP::alloc(buffer->cdata(), siglen);
IPP::zero_mem(buffer->cdata(), siglen);
\end{lstlisting}
Come si pu\`o vedere, l'allocazione della memoria veniva effettuata manualmente
dopo l'inizializzazione dell'oggetto, mentre non era presente alcuna
deallocazione della memoria. Per risolvere questo problema ci sono due strade
possibili: o si aggiunge una deallocazione della memoria dopo che i dati sono
stati scritti su output, o si sposta il meccanismo di allocazione/deallocazione
all'interno del costruttore/distruttore della classe. Per rendere l'utilizzo di
\texttt{FFTBuf} pi\`u trasparente ed evitare che il problema si ripeta se questa
classe dovesse essere riutilizzata da qualche altra parte, la scelta \`e
ricaduta sulla seconda soluzione.

\subsubsection{Implementazione corretta}
La nuova versione dell'allocazione di  memoria per il buffer interno a
\texttt{FFTBuf} \`e la seguente:
\begin{lstlisting}
template<class T> FFTBuf<T>::FFTBuf(long int siglen) : _dst(NULL),
    _siglen(siglen), _expected_sums(1), _processed_sums(0),
    _assigned_sources(0), _written(false)
{
    _dst = IPP::alloc(_dst, siglen);
    IPP::zero_mem(_dst, siglen);
}

template<class T> FFTBuf<T>::FFTBuf(long int siglen, int sums) :
    _dst(NULL), _siglen(siglen), _expected_sums(sums),
    _processed_sums(0), _assigned_sources(0), _written(false)
{
    _dst = IPP::alloc(_dst, siglen);
    IPP::zero_mem(_dst, siglen);
}

template<class T> FFTBuf<T>::~FFTBuf()
{
    notify_all();
    IPP::free(_dst);
    _dst = NULL;
}
\end{lstlisting}
Ora l'allocazione \`e legata alla creazione di un nuovo oggetto di tipo
\texttt{FFTBuf} e la deallocazione viene effettuata automaticamente quando
l'oggetto viene distrutto: a questo modo si ha la certezza che tutti i buffer
creati vengono distrutti correttamente e che non ci sar\`a perdita di memoria.
Dopo questa modifica, il programma \`e passato da pochi minuti di operativit\`a
con consumo di memoria sempre crescente a un consumo di memoria stabile e
conseguente stabilit\`a del programma stesso.

\section{Programmi di supporto}
Allo scopo di testare l'applicazione e verificarne il funzionamento, sono stati
sviluppati alcuni programmi di supporto. Trattandosi di programmi in cui le
prestazioni non sono essenziali, il linguaggio di programmazione utilizzato \`e
Python per favorire uno sviluppo pi\`u rapido.
\subsection{Server UDP}
Uno dei programmi sviluppati serve ad inviare dei dati nel formato corretto
tramite il protocollo UDP. Siccome l'ordinamento \`e importante,
anche se non \`e grave l'eventuale perdita di alcuni pacchetti, all'inizio di
ogni pacchetto UDP compare un contatore progressivo lungo 4 byte. Il client
controlla questo contatore e se il numero non \`e in progressione con il
pacchetto precedente, sa quanti pacchetti ha perso e svuota i buffer
eventualmente pieni per ricominciare ad elaborare i dati a partire dal nuovo
pacchetto ricevuto. Questa operazione viene svolta per evitare di elaborare dati
incompleti che potrebbero compromettere i risultati. Il server in Python
permette di leggere i dati da un file ed inviarli tramite rete ad una
determinata destinazione.
\subsection{Visualizzatore di grafici}
\label{pyvis}
Un altro programma di supporto utlissimo per verificare il funzionamento del
programma principale \`e il visualizzatore di grafici. Questo programma prende i
dati elaborati e visualizza un grafico in scala logaritmica(?). La scala
logaritmica viene utilizzata in quanto permette di visualizzare i dati in
maniera pi\`u facilmente comprensibile rispetto ad una scala naturale, senza
cambiarne il significato. Visualizzare i dati \`e un modo semplice per
verificare la correttezza delle trasformazioni, se il segnale trasformato \`e
noto.
