\chapter{Sviluppo del progetto}
\label{sw_devel}

Durante lo sviluppo del progetto si ha fatto uso di librerie e strumenti
software per semplificare il pi\`u possibile il lavoro da svolgere ed ottenere
il massimo delle prestazioni possibili. 

\section{Librerie}
L'utilizzo delle giuste librerie permette di sviluppare pi\`u velocemente i
propri progetti, ottenere prestazioni spesso migliori rispetto ad una soluzione
sviluppata in proprio e sicuramente avere pi\`u certezze riguardo al corretto
funzionamento delle parti attenenti alla libreria.

\subsection{Boost}
Le librerie Boost\footnote{\url{http://www.boost.org/}} sono molto usate nella
programmazione in \CC perch\'e implementano moltissime funzioni spesso usate
fornendo una interfaccia molto semplice. L'ottima qualit\`a di queste librerie
\`e confermato dal fatto che spesso alcune sue componenti vengono integrate
nello standard \CC. In questo progetto vengono utilizzate principalmente per
semplificare la lettura di argomenti da linea di comando, la lettura/scrittura
di files e il multithreading.

\subsection{\ac{ipp}}
Le Integrated Performance
Primitives\footnote{\url{http://software.intel.com/en-us/intel-ipp/}} sono delle
librerie che implementano un insieme molto completo di funzioni e algoritmi
relativi al processamento di segnali e crittografia. In particolare implementano
funzioni utili al processamento di segnali audio (filtri, codifiche audio,
compressione dei dati, ecc.), funzioni per il processamento di immagini
(trasformazioni, codifica video, ecc.), funzioni per calcoli su matrice e
funzioni crittografiche (crittografia simmetrica, asimmetrica, funzioni hash,
ecc.).  Il vantaggio delle IPP \`e che offrono un ampio spettro di funzioni che
coprono praticamente tutte le necessit\`a nel processamento dei segnali, oltre
ad essere ottimizzate per processori Intel. Inoltre sono state scritte in modo
da essere naturalmente utilizzabili in ambienti multithreaded, sfruttando tutte
le potenzialit\`a di processori multicore o processori multipli. Tutti questi
fattori hanno concorso nell'adozione di questa libreria al posto di altre
librerie concorrenti. Questa libreria \`e l'unico componente non Open Source
utilizzato nell'intero progetto, anche se il suo utilizzo \`e gratuito per usi
accademici. Le librerie alternative potrebbero essere comparate per verificare
l'effettiva qualit\`a a livello di prestazioni.

\section{Codice pre-esistente}
Parte del codice del progetto \`e stato prelevato dal software sviluppato in
fase di tirocinio, sempre presso l'Istituto di Radioastronomia di Medicina.
Questo codice ha permesso di ridurre notevolmente il tempo dedicato allo
sviluppo del calcolo della \ac{FFT} lasciando maggior tempo per una corretta
implementazione delle altre parti. Questo codice \`e stato adattato per
funzionare in un contesto multithreaded e per accettare input da diverse fonti.

\section{Strumenti di supporto}
Per semplificare lo sviluppo esistono diversi software che sono piuttosto
standard in un qualunque progetto: l'editor di testo, il compilatore, un sistema
di versioning e un debugger.

\subsection{Editor}
La selezione dell'editor dipende principalmente dalle preferenze del
programmatore ed in minima parte dalle feature che questi offre. Ambienti che
integrano l'editor all'interno di un'interfaccia che permette di gestire anche
altri aspetti, come la compilazione, vengono chiamati \ac{IDE}. Nell'ambito di
questo progetto l'unico editor utilizzato \`e stato
VIM\footnote{\url{http://www.vim.org/}}.

\subsection{Compilatore}
La scelta del compilatore dipende in prima istanza dal linguaggio di
programmazione prescelto. Nel caso di questo progetto il linguaggio di
programmazione utilizzato \`e il \CC, scelto per le sue ottime prestazioni,
oltre alla possibilit\`a di utilizzare il paradigma di programmazione
orientata agli oggetti. Come compilatore \`e stato usato lo standard su sistemi
unix, cio\`e g++. Questo compilatore fa parte della suite di compilatori del
progetto GNU\footnote{\url{http://www.gnu.org/}}.

\subsection{Versioning Control System}

Un software per il controllo delle revisioni serve a facilitare lo sviluppo di
uno stesso progetto da parte di diverse persone, oltre a fornire un modo pratico
di salvare le vecchie versioni del codice. Anche nel caso di sviluppatore
singolo, un sistema di versioning permette lo sviluppo su macchine diverse con
la certezza che non si lavori su del codice non aggiornato.

\subsubsection{Subversion}
Subversion\footnote{\url{http://subversion.tigris.org/}} \`e uno dei VCS pi\`u
diffusi, \`e basato su architettura client/server dove il server si occupa di
salvare i dati e le informazioni sulla storia del progetto, mentre i client
vengono usati dagli sviluppatori per recuperare queste informazioni. Il progetto
\`e salvato su un repository SVN localizzato su un server dell'Istituto di
Radioastronomia.

\subsubsection{Git}
Git\footnote{\url{http://git-scm.com/}} \`e un altro sistema di versioning che,
a differenza di subversion, si basa su un modello distribuito: quando uno
sviluppatore recupera le informazioni del progetto, queste vengono salvate tutte
in locale. Questo significa che \`e possibile sviluppare e modificare il
repository anche se non connessi alla rete, e quindi al server centrale, cosa
non possibile con subversion. Git offre anche prestazioni migliori e una serie
di altri vantaggi rispetto a subversion. Un grande vantaggio di git \`e la sua
possibilit\`a di lavorare su un repository di subversion, comportandosi quindi
come un client di subversion nei confronti del server, ma offrendo tutti i
vantaggi di git allo sviluppatore. Questo sistema di versioning \`e stato
utilizzato dal lato sviluppatore, sincronizzandosi con il server SVN.

\subsection{Autotools}
Gli autotools sono lo strumento di gestione progetti più diffuso nel software
open source. Anche se a volte un po' complicati da usare, la semplificazione
della gestione del progetto rendono il loro utilizzo indispensabile: con un
solo comando, infatti, \`e possibile compilare, testare e creare un archivio del
progetto pronto per essere redistribuito. \`E importante ricordarsi che questi
tool tendono a cambiare comportamento tra le singole versioni, quindi va sempre
controllato se siano necessari aggiornamenti ai files di configurazione.

\subsection{Strumenti di debugging}
Per individuare eventuali errori nel codice esistono degli strumenti in grado di
aiutare il programmatore a scoprirli e rimuoverli.

\subsubsection{Debugger}
Il debugger \`e uno strumento utilissimo per cercare la provenienza di errori
fatali, in particolare le segmentation faults, e permette di tracciare
l'esecuzione del programma rispetto al codice sorgente. Il debugger standard per
sistemi Unix \`e gdb, sempre parte del progetto GNU. Il debugger \`e stato
utilizzato principalmente per capire lo stato dei vari thread ed indagare i
motivi per cui potevano accadere alcuni particolari eventi, come un deadlock o
una segmentation fault.

\subsubsection{Valgrind}
La suite di tools chiamata Valgrind \`e incentrata sull'individuazione di
memory leak; il suo utilizzo \`e stato cruciale per individuare e rimuovere gli
sprechi di memoria e capire sotto quali condizioni non venisse rilasciata la
memoria utilizzata.
