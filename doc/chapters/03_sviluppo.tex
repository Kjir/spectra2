\chapter{Sviluppo del progetto}
\label{sw_devel}

Durante lo sviluppo del progetto si ha fatto uso di librerie e strumenti
software per semplificare il pi\`u possibile il lavoro da svolgere ed ottenere
il massimo delle prestazioni possibili. 

\section{Librerie}
L'utilizzo delle giuste librerie permette di sviluppare pi\`u velocemente i
propri progetti, ottenere prestazioni spesso migliori rispetto ad una soluzione
sviluppata in proprio e sicuramente avere pi\`u certezze riguardo al corretto
funzionamento delle parti attenenti alla libreria.

\subsection{Boost}
Le librerie Boost\footnote{\url{http://www.boost.org/}} sono molto usate nella
programmazione in \CC perch\'e implementano moltissime funzioni di comune
utilizzo fornendo una interfaccia molto semplice. L'ottima qualit\`a di queste
librerie \`e confermato dal fatto che spesso alcune sue componenti vengono
integrate nello standard \CC. Queste librerie sono basate sulla versione ANSI
del \CC, rendendole compatibili con qualunque compilatore che implementi gli
standard ANSI.In questo progetto vengono utilizzate principalmente per
semplificare la lettura di argomenti da linea di comando, la lettura/scrittura
di files, le comunicazioni di rete e il multithreading.

\subsection{\ac{ipp}}
Le Integrated Performance
Primitives\footnote{\url{http://software.intel.com/en-us/intel-ipp/}} sono delle
librerie che implementano un insieme molto completo di funzioni e algoritmi
relativi al processamento di segnali e crittografia. In particolare implementano
funzioni utili al processamento di segnali audio (filtri, codifiche audio,
compressione dei dati, ecc.), funzioni per il processamento di immagini
(trasformazioni, codifica video, ecc.), funzioni per calcoli su matrice e
funzioni crittografiche (crittografia simmetrica, asimmetrica, funzioni hash,
ecc.).  Il vantaggio delle IPP \`e che offrono un ampio spettro di funzioni che
coprono praticamente tutte le necessit\`a nel processamento dei segnali, oltre
ad essere ottimizzate per processori Intel. Inoltre sono state scritte in modo
da essere naturalmente utilizzabili in ambienti multithreaded, sfruttando tutte
le potenzialit\`a di processori multicore o processori multipli. Tutti questi
fattori hanno concorso nell'adozione di questa libreria al posto di altre
librerie concorrenti. Questa libreria \`e l'unico componente non Open Source
utilizzato nell'intero progetto, anche se il suo utilizzo \`e gratuito per usi
accademici. Le librerie alternative potrebbero essere comparate per verificare
l'effettiva qualit\`a a livello di prestazioni\footnote{cfr. paragrafo
\ref{altlib}}.

\subsubsection{Multithreading e prestazioni nelle \ac{ipp}}
Le \ac{ipp} sono state sviluppate con l'intenzione di sfruttare a pieno le
istruzioni \ac{SIMD} e \ac{SSE} presenti nei moderni processori. Queste
istruzioni permettono di effettuare operazioni su vettori in parallelo,
guadagnando grandi vantaggi prestazionali in alcuni tipi di operazioni, tra cui
anche i calcoli necessari per l'elaborazione di segnali. Siccome diversi
processori supportano diversi tipi di istruzioni, possono esistere diverse
implementazioni di alcune funzione in base al tipo di istruzioni supportate. Per
selezionare automaticamente la funzione corretta, le \ac{ipp} dispongono di un
dispatcher che, durante l'inizializzazione a run-time, individuano le capacit\`a
del processore e quindi la categoria di libreria da utilizzare. Questo permette
di usare trasparentemente le funzioni di libreria sfruttando sempre
l'implementazione pi\`u efficiente per il processore utilizzato.

La libreria \`e strutturata su funzioni di basso livello, le quali costituiscono
la base della libreria, che 


\section{Codice pre-esistente}
Parte del codice del progetto \`e stato prelevato dal software sviluppato in
fase di tirocinio, sempre presso l'Istituto di Radioastronomia di Medicina.
Questo codice ha permesso di ridurre notevolmente il tempo dedicato allo
sviluppo del calcolo della \ac{FFT} lasciando maggior tempo per una corretta
implementazione delle altre parti. Questo codice \`e stato adattato per
funzionare in un contesto multithreaded e per accettare input da diverse fonti.

\section{Strumenti di supporto}
Per semplificare lo sviluppo esistono diversi software che sono piuttosto
standard in un qualunque progetto: l'editor di testo, il compilatore, un sistema
di versioning e un debugger.

\subsection{Compilatore}
La scelta del compilatore dipende in prima istanza dal linguaggio di
programmazione prescelto. Nel caso di questo progetto il linguaggio di
programmazione utilizzato \`e il \CC, scelto per le sue ottime prestazioni,
oltre alla possibilit\`a di utilizzare il paradigma di programmazione
orientata agli oggetti. Come compilatore \`e stato usato lo standard su sistemi
unix, cio\`e g++. Questo compilatore fa parte della suite di compilatori del
progetto GNU\footnote{\url{http://www.gnu.org/}}.

\subsection{Strumenti di debugging}
Per individuare eventuali errori nel codice esistono degli strumenti in grado di
aiutare il programmatore a scoprirli e rimuoverli.

\subsubsection{Debugger}
Il debugger \`e uno strumento utilissimo per cercare la provenienza di errori
fatali, in particolare le segmentation faults, e permette di tracciare
l'esecuzione del programma rispetto al codice sorgente. Il debugger standard per
sistemi Unix \`e gdb, sempre parte del progetto GNU. Il debugger \`e stato
utilizzato principalmente per capire lo stato dei vari thread ed indagare i
motivi per cui potevano accadere alcuni particolari eventi, come un deadlock o
una segmentation fault.

\subsubsection{Valgrind}
La suite di tools chiamata Valgrind \`e incentrata sull'individuazione di
memory leak; il suo utilizzo \`e stato cruciale per individuare e rimuovere gli
sprechi di memoria e capire sotto quali condizioni non venisse rilasciata la
memoria utilizzata.
