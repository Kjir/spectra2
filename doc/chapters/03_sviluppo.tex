\chapter{Sviluppo del progetto}
\label{sw_devel}

Per lo sviluppo del progetto sono state utilizzati una serie di strumenti e
librerie open source, in modo di semplificare e velocizzare lo sviluppo del
software. Tutte le fasi di sviluppo sono state eseguite su sistema operativo
Linux.
\section{Strumenti di supporto}
Per semplificare lo sviluppo esistono diversi software che sono piuttosto
standard in un qualunque progetto: l'editor di testo, il compilatore, un sistema
di versioning.

\subsection{Editor}
La selezione dell'editor dipende principalmente dalle preferenze del
programmatore ed in minima parte dalle feature che questi offre. Ambienti che
integrano l'editor all'interno di un'interfaccia che permette di gestire anche
altri aspetti, come la compilazione, vengono chiamati \ac{IDE}. Nell'ambito di
questo progetto l'unico editor utilizzato \`e stato
VIM\footnote{\url{http://www.vim.org/}}.

\subsection{Compilatore}
La scelta del compilatore dipende in prima istanza dal linguaggio di
programmazione prescelto. Nel caso di questo progetto il linguaggio di
programmazione utilizzato \`e il \CC, scelto per le sue ottime prestazioni,
oltre alla possibilit\`a di utilizzare il paraadigma di programmazione
orientata agli oggetti. Come compilatore \`e stato usato lo standard su sistemi
unix, cio\`e g++. Questo compilatore fa parte della suite di compilatori del
progetto GNU.

\subsection{Versioning Control System}
\subsubsection{Subversion}

\subsubsection{Git}

\subsection{Autotools}
Gli autotools sono lo strumento di gestione progetti più diffuso nel software
open source. Anche se a volte un po' complicati da usare, la semplificazione
della gestione del progetto rendono il loro utilizzo indispensabile: con un
solo comando, infatti, \`e possibile compilare, testare e creare un archivio del
progetto pronto per essere redistribuito. \`E importante ricordarsi che questi
tool tendono a cambiare comportamento tra le singole versioni, quindi va sempre
controllato se siano necessari aggiornamenti ai files di configurazione.

\section{Librerie}

\subsection{Boost}
Le librerie boost sono molto usate nella programmazione in \CC perch\'e
implementano moltissime funzioni spesso usate fornendo una interfaccia molto
semplice. L'ottima qualit\`a di queste librerie \`e confermato dal fatto che spesso
alcune sue componenti vengono integrate nello standard \CC. In questo progetto
vengono utilizzate principalmente per semplificare la lettura di argomenti da
linea di comando, la lettura/scrittura di files e il multithreading.

\subsection{\ac{ipp}}
Le Integrated Performance Primitives sono delle librerie che implementano un
insieme molto completo di funzioni e algoritmi relativi al processamento di
segnali e crittografia. In particolare implementano funzioni utili al
processamento di segnali audio (filtri, codifiche audio, compressione dei dati,
ecc.), funzioni per il processamento di immagini (trasformazioni, codifica
video, ecc.), funzioni per calcoli su matrice e funzioni crittografiche
(crittografia simmetrica, asimmetrica, funzioni hash, ecc.).  Il vantaggio
delle IPP \`e che offrono un ampio spettro di funzioni che coprono praticamente
tutte le necessit\`a nel processamento dei segnali, oltre ad essere ottimizzate
per processori Intel. Inoltre sono state scritte in modo da essere naturalmente
utilizzabili in ambienti multithreaded, sfruttando tutte le potenzialit\`a di
processori multicore o processori multipli. Tutti questi fattori hanno concorso
nell'adozione di questa libreria al posto di altre librerie concorrenti. Questa
libreria \`e l'unico componente non Open Source utilizzato nell'intero progetto,
anche se il suo utilizzo \`e gratuito per usi accademici. Le librerie
alternative potrebbero essere comparate per verificare l'effettiva qualit\`a a
livello di prestazioni.

\section{Codice pre-esistente}


