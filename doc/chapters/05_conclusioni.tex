\chapter{Conclusioni e sviluppi futuri}
\label{conclusions}
\section{Conclusioni}
Lo scopo principale dello sviluppo di questo programma \`e capire che tipo di
prestazioni ci si pu\`o aspettare da una soluzione software sull'elaborazione di
segnali radioastronomici. Basandosi sui risultati dei test, si vede che il
programma \`e in grado di elaborare correttamente delle quantit\`a importanti di
dati; se confrontate con le prestazioni di una macchina DSP\footnote{DSP
    significa, in questo caso, Digital Signal Processor. Con questo nome si
    indicano degli elaboratori specializzati che effettuano solamente le
    trasformate per cui sono costruiti.} sicuramente saranno inferiori, ma se si
aggiunge al confronto anche i fattori del costo dell'attrezzatura e la
flessibilit\`a della soluzione alle esigenze si capisce come le prestazioni di
questo software siano importanti: il software lavora su un normale elaboratore
il cui prezzo, nel caso di macchine comunque high-end e piuttosto potenti, si
aggira attorno ai 6000 euro. L'hardware specializzato per il DSP ha costi molto
pi\`u elevati, nell'ordine dei 10000 euro, anche se spesso i consumi sono
minori. Inoltre, modificare un programma scritto in \CC\, \`e un'operazione
relativamente semplice e che molte persone sono in grado di fare, mentre per
modificare l'algoritmo in una macchina DSP bisogna modificare l'hardware,
operazione per nulla semplice, n\'e rapida, quindi anche pi\`u costosa. Avere
una soluzione un po' meno performante, ma comunque di buon livello a costi
contenuti \`e importante sia in quei campi di ricerca dove le prestazioni non
sono un aspetto fondamentale del lavoro, sia per progetti a budget molto
ristretto, dove avere degli strumenti a basso costo \`e fondamentale.
Un altro aspetto importante \`e il fatto che un elaboratore generico pu\`o
essere riutilizzato per altri scopi, ad esempio alla fine del progetto o nei
tempi ``morti'' in cui non serve fare elaborazioni sui segnali, mentre hardware
specifico per il DSP \`e difficile da riutilizzare in altri ambiti.

In generale, dipende molto dalla natura della ricerca e delle analisi da
effettuare: in alcuni casi una soluzione software \`e pi\`u conveniente, in
altri il costo iniziale di un DSP viene ammortizzato nel tempo. Va considerato,
anche, che difficilmente una soluzione software pu\`o esulare completamente
dall'utilizzo di un DSP, ma pu\`o ridurne comunque il numero o aumentare il tipo
di elaborazioni fattibili sui dati pre-elaborati dal DSP. I risultati raggiunti
mostrano come la soluzione software possa essere affiancata, o in alcuni casi
addirittura sostituita, ai tradizionali DSP per offire una soluzione pi\`u
adeguata alle esigenze della ricerca. Come in tanti altri casi, non c'\`e una
soluzione unica per risolvere tutti i problemi, ma questo spettrometro \`e
sicuramente uno strumento addizionale a disposizione del ricercatore.

\section{Sviluppi futuri}
Gli ottimi risultati conseguiti aprono la strada a diversi sviluppi possibili:
si può adattare il programma ad altri scopi, provare implementazioni diverse,
cercare di migliorarne ancora le prestazioni e tanto altro ancora.
\subsection{Estensione del programma}

\subsubsection{Implementazione di altre trasformazioni}
Il modo pi\`u semplice per estendere le funzionalit\`a del programma \`e
scrivere qualche trasformazione o filtro aggiuntivo da accodare nella
\texttt{FilterChain}. Essendo gi\`a pronte le interfacce, ed esistendo gi\`a una
sua implementazione, la parte di lavoro necessaria riguarda solamente la
scrittura delle funzioni di trasformazione, senza dover sviluppare un modo per
gestire il flusso dei dati.

\subsubsection{Selezione dinamica del tipo di dato}
\label{dyn_dtype}
Nell'implementazione attuale, la selezione del tipo di dati in ingresso ed in
uscira \`e basato su delle costanti all'interno del codice. Per modificare
queste costanti bisogna modificare il programma e ricompilarlo, operazione che,
per quanto semplice, potrebbe essere rimossa completamente permettendo di
selezionare il tipo di dato ad esempio da linea di comando.

\subsubsection{Controllo interattivo dei parametri}
Un altro possibile sviluppo del programma \`e la possibilit\`a di controllare in
modo interattivo i parametri del programma ed avviare/fermare il processo
tramite una console accessibile tramite rete. Il funzionamento \`e piuttosto
semplice: connettendosi tramite rete alla console, si possono impartire vari
comandi per modificare il tipo di trasformazioni effettuate ed i loro parametri,
la fonte di acquisizione e la destinazione dei dati. Si può anche avviare e
fermare il ciclo principale di elaborazione e leggere delle statistiche sul suo
funzionamento.

\subsubsection{Output di rete}
Attualmente l'unica implementazione del \texttt{SinkFilter} \`e una classe che
scrive l'output su file. Molto utile potrebbe essere la scrittura su una
interfaccia di rete, cos\`i che i dati possano essere letti da un computer
remoto. Questo permetterebbe anche di visualizzare i dati in tempo reale
spostando sul client il programma di visualizzazione.

\subsubsection{Acquisizione dati \ac{seti}}
\label{seti}
L'acquisizione di dati per il progetto \ac{seti} non avviene tramite rete, ma
utilizza uno speciale hardware dedicato. Leggere i dati da questo hardware
richiede un certo lavoro via software che potrebbe essere integrato all'interno
dell'interfaccia \texttt{SourceFilter}, cos\`i che il programma possa essere
utilizzato per elaborare i dati per il progetto \ac{seti}.

\subsection{Librerie alternative}
\label{altlib}

Esistono diverse liberie alternative alle \ac{ipp} che svolgono le stesse
operazioni. Per avere parametri di confronto, sarebbe interessante utilizzarle
nello stesso tipo di programma per verificare se effettivamente le
ottimizzazioni presenti nelle \ac{ipp} offrono grandi vantaggi rispetto ad
implementazioni alternative.

\subsubsection{FFTW: The Fastest Fourier Transform in the West}
La libreria FFTW \`e un progetto Open Source che sostiene di avere la pi\`u
rapida implementazione della \ac{FFT}.

\subsubsection{Framewave}
La libreria Open Source Framewave \`e stata sviluppata da AMD per fare
concorrenza alle \ac{ipp}. L'interfaccia di Framewave \`e molto simile a quella
delle \ac{ipp}, rendendo quindi pi\`u semplice l'adattamento del programma
all'utilizzo di questa libreria.

\subsubsection{\ac{FFT} sulla GPU}
Esistono diverse librerie che invece di sfruttare il processore per calcolare la
\ac{FFT}, utilizzano la GPU della scheda grafica. Alcuni
benchmark\footnote{\url{http://www.cv.nrao.edu/~pdemores/gpu/}} mostrano un
miglioramento interessante delle prestazioni rispetto a librerie per CPU,
soprattutto con l'aumentare il numero di canali. Esistono diverse librerie che
permettono di sfruttare la GPU, tra cui
CUDA\footnote{\url{http://www.nvidia.com/object/cuda_home_new.html}} e
GPUFFTW\footnote{\url{http://gamma.cs.unc.edu/GPUFFTW/}}. Se queste librerie si
dovessero mostrare realmente efficaci, i requisiti hardware si limiterebbero
fondamentalmente alla scheda grafica, permettendo di ridurre ulteriormente il
costo di una soluzione software.

\subsection{Compilatori}
Un altro aspetto da esplorare sono i compilatori: nel progetto \`e stato
utilizzato il compilatore g++, parte della suite di compilatori GNU gcc, con le
sue opzioni standard. Siccome dalla qualit\`a del compilatore e dalle opzioni
utilizzate dipende anche la qualit\`a del codice macchina generato e quindi la
sua rapidit\`a, un miglioramento di prestazioni potrebbe essere possibile
esplorando diverse opzioni.

\subsubsection{GCC/g++}
Il compilatore GNU gcc, nella sua versione per il \CC\, g++, \`e il pi\`u
diffuso in ambiente *NIX. Supporta numerose architetture e per ogni architettura
ha delle specifiche opzioni per ottimizzare il codice prodotto. Siccome nel
progetto \`e stato utilizzato senza particolari opzioni, sarebbe interessante
verificare se \`e possibile spremere prestazioni migliori con l'utilizzo delle
funzionalit\`a avanzate di g++.

\subsubsection{ICC}
Il compilatore ICC \`e stato sviluppato da Intel ed \`e il compilatore con cui
sono state compilate le liberie \ac{ipp} stesse. Essendo stato sviluppato
dall'azienda produttrice di processori, potrebbe essere in grado di sfruttare
molto meglio l'architettura dei processori Intel rispetto ad altri compilatori.

\subsubsection{Clang e LLVM}
Clang e LLVM formano insieme un compilatore e la sua interfaccia; sono un
progetto relativamente giovane e sono stati sviluppati da Apple con una licenza
di tipo BSD. Questo nuovo compilatore sembra produrre codice molto più rapido
rispetto gli altri compilatori e sta guadagnando un seguito molto importante
nella comunit\`a FreeBSD e non solo.

\subsection{Altri sviluppi}
La natura modulare del programma lo prestano ad un'infinit\`a di manipolazioni
ed utilizzi nuovi. Con i dovuti arrangiamenti potrebbe addirittura essere
utilizzato per scopi completamente differenti, come ad esempio la codifica di
file audio o l'elaborazione di immagini. A seconda delle necessit\`a si possono
apportare piccoli miglioramenti e l'architettura pensata cerca di essere il
pi\`u possibile flessibile verso esigenze diverse. Come per ogni progetto
software, le vere potenzialit\`a ed i veri limiti diventeranno evidenti
solamente nel tempo e tentando di utilizzare il programma negli ambiti pi\`u
differenti.
