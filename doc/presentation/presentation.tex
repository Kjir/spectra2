\documentclass{beamer}

\usepackage[italian]{babel}

\mode<presentation>{
    \usetheme[left]{Marburg}
    \useinnertheme[shadow]{rounded}
}

\AtBeginSubsection[]
{
  \begin{frame}<beamer>{Contenuti}
    \tableofcontents[currentsection,currentsubsection]
  \end{frame}
}

\title[Software per segnali radioastronomici]{Sviluppo di un software per l'analisi real-time di dati
radioastronomici su macchine multicore}
\author{St\'ephane Bisinger}

\begin{document}

\begin{frame}
\titlepage
\end{frame}
\begin{frame}{Riassunto}
\tableofcontents
\end{frame}
\section{Teoria}
\subsection{Analisi dei segnali}
\begin{frame}{Cosa sono i segnali?}
I segnali sono bla bla bla.

Ad esempio:
\begin{itemize}
    \item Suoni (pressione acustica su spazio)\pause
    \item Immagini (concentrazione di colore su coordinate spaziali)\pause
    \item Video (come le immagini, solo con in pi\`u la dimensione del tempo)
\end{itemize}
\end{frame}

\begin{frame}{Come si analizza un segnale?}
\begin{itemize}
    \item Un qualunque segnale pu\`o essere espresso come somma pesata di
    sinusoidi. \pause
    \item Quindi un segnale pu\`o essere espresso in relazione alla frequenza
    delle sinusoidi parte del segnale \pause $->$ Rappresentazione del segnale nel
    dominio delle frequenze.
\end{itemize}
\end{frame}

\subsection{Radioastronomia}
\begin{frame}{I radiotelescopi}
Un radiotelescopio osserva le onde elettromagnetiche provenienti dallo spazio
con frequenza inferiore a 3 Ghz.
\end{frame}
\section{Il software}
\subsection{La struttura}
\subsection{Threading}
\section{Risultati e Conclusioni}
\subsection{Risultati dei test}
\subsection{Conclusioni}
\subsection{Sviluppi futuri}
\end{document}
